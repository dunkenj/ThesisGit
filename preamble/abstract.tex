%
%       --- abstract.tex ---
%
% Should be of the order of 300 words.

%\newpage
\chapter*{Abstract}

This thesis explores the growth of galaxies during the first few billion years of galaxy formation and their potential role as the sources which powered the process of reionization. The data used throughout the thesis is taken from the Cosmic Assembly Near-infrared Extragalactic Legacy Survey (CANDELS).

First, we measure new estimates for the galaxy stellar mass function and star formation rates for samples of galaxies at $z \sim 4,~5,~6~\&~7$ using data in the CANDELS GOODS South field. The deep near-infrared observations allow us to construct the stellar mass function at $z \geq 6$ directly for the first time. We estimate stellar masses for our sample by fitting the observed spectral energy distributions with synthetic stellar populations, including the contributions from nebular line and continuum emission. The observed UV luminosity functions for the samples are consistent with previous observations, however we find that the observed $M_{UV}$ - M$_{*}$ relation has a shallow slope more consistent with a constant mass to light ratio and a normalisation which evolves with redshift. 

We observe stellar mass functions which have steep low-mass slopes ($\alpha \approx -1.9$), steeper than previously observed at these redshifts and closer to that of the UV luminosity function. Integrating our new mass functions, we find the observed stellar mass density evolves from $\log_{10} \rho_{*} = 6.64^{+0.58}_{-0.89}$ at $z \sim 7$ to  $7.36\pm0.06$ $\text{M}_{\odot} \text{Mpc}^{-3}$ at $z \sim 4$. Combining the measured UV continuum slopes ($\beta$) with their rest-frame UV luminosities, we calculate dust corrected star-formation rates (SFR) for our sample. We find the specific star-formation rate for a fixed stellar mass increases with redshift whilst the global SFR density falls rapidly over this period. Our new SFR density estimates are higher than previously observed at this redshift.

Next, we utilise the same dataset to test a new method for estimating the merger fraction of galaxies in photometric surveys. Using a probabilistic method for estimating close galaxy pairs using photometric redshift probability distributions, we estimate the merger fraction of galaxies at $z\geq 2$. For projected separations of $5 \leq r_{\text{p}} \leq 20$ kpc and $5 \leq r_{\text{p}} \leq 30$ kpc we measure the merger fraction for mass selected samples of $\log_{10}(\Mstar/\text{M}_{\odot}) > 9.5$ and $\log_{10}(\Mstar/\text{M}_{\odot}) > 10$ and merger ratios of 1:4 or less (major mergers).

For assumed merger timescales based on hydrodynamical simulations, we estimate the average time between mergers per galaxy ($\Gamma$, Gyr) and the comoving merger rate ($\mathcal{R}$, $\text{Gyr}^{-1}~\text{Mpc}^{-3}$). Over the redshift range $2 \lesssim z \lesssim 4$ we find that the average time between mergers per galaxy is approximately constant at $\approx 4$ Gyr ($\log_{10}(\Mstar/\text{M}_{\odot}) > 9.5$) or $\approx 8$ Gyr ($\log_{10}(\Mstar/\text{M}_{\odot}) > 10$). Compared to the star-formation rates measured for galaxies at these masses, we conclude that star-formation is the dominant form of growth (by a factor $\approx ~10\times$) during this epoch. Although we find that the methodology performs well at $z\lesssim 4$, more data is required to make robust estimations of the merger fraction at $z\sim5$ or $z\sim6$. Similarly, tighter constraints on the observed stellar mass functions are required before we can draw meaningful conclusions from the observed comoving merger rates.

Finally, we present a new analysis of the ionizing emissivity ($\dot{N}_{\rm{ion}}$, s$^{-1}$ Mpc$^{-3}$) for galaxies during the epoch of reionization and their potential for completing and maintaining reionization. We use extensive SED modelling -- incorporating two plausible mechanisms for the escape of Lyman continuum photon -- to explore the range and evolution of ionizing efficiencies consistent with new results on galaxy colours ($\beta$) during this epoch. We estimate $\dot{N}_{\rm{ion}}$ for the latest observations of the luminosity and star-formation rate density at $z<10$, outlining the range of emissivity histories consistent with our new model. 

Given the growing observational evidence for a UV colour-magnitude relation in high-redshift galaxies, we find that for any plausible evolution in galaxy properties, red (brighter) galaxies are less efficient at producing ionizing photons than their blue (fainter) counterparts. The assumption of a redshift and luminosity evolution in $\beta$ leads to two important conclusions. Firstly, the ionizing efficiency of galaxies naturally increases with redshift.
Secondly, for a luminosity dependent ionizing efficiency, we find that galaxies down to a rest-frame magnitude of $M_{\rm{UV}} \approx -15$ alone can potentially produce sufficient numbers of ionizing photons to maintain reionization as early as $z\sim8$ for a clumping factor of $C_{\textsc{Hii}} \leq 3$.





\clearpage



%%
%% End of file...
%%

