%%
%%       --- a chapter ---
%%
%% A chapter

\chapter[Data]{Long chapter heading} 
\label{ch:chapter2_label}

\section[Sec]{Section}
\label{sec:section2_label}

The photometry used throughout this work is taken from the catalog of \citet{Guo:2013ig}, a UV to mid-infrared multi-wavelength catalog in the CANDELS GOODS South field based on the CANDELS WFC3/IR observations combined with existing public data.

\subsection{Imaging Data}
The near-infrared WFC3/IR data combines observations from the CANDELS survey \citep{2011ApJS..197...35G,Koekemoer:2011br} with the WFC3 Early Release Science (ERS; \citeauthor{2011ApJS..193...27W}~\citeyear{2011ApJS..193...27W}) and Hubble Ultra Deep Field (HUDF; PI Illingworth; \citeauthor{Bouwens:2010dk}~\citeyear{Bouwens:2010dk}) surveys. The southern two thirds of the field (incorporating the CANDELS `DEEP' and `WIDE' regions and the UDF) were observed in the F105W, F125W and F160W bands. The northern-most third, comprising the ERS region, was observed in F098M, F125W and F160W. In addition to the initial CANDELS observations, the GOODS South field was also observed in the alternative J band filter, F140W, as part of the 3D-HST survey (Brammer et al. 2012).

The optical HST images from the Advanced Camera for Surveys (ACS) images are version v3.0 of the mosaicked images from the GOODS HST/ACS Treasury Program, combining the data of \citet{2004ApJ...600L..93G} with the subsequent observations obtained by \citet{2006AJ....132.1729B} and \citep{Koekemoer:2011br}. The field was observed in the F435W, F606W, F775W, F814W and F850LP bands. Throughout the paper, we will refer to the HST filters F435W, F606W, F775W, F814W, F850LP, F098M, F105W, F125W, F160W as $B_{435}$, $V_{606}$, $i_{775}$, $I_{814}$, $z_{850}$, $Y_{098}$, $Y_{105}$, $J_{125}$, $H_{160}$ respectively. 

The \emph{Spitzer}/IRAC \citep{Fazio:2004eb} 3.6 and 4.5$\mu m$ images were taken from the Spitzer Extended Deep Survey (PI: G. Fazio, \citeauthor{Ashby:2013cc}~\citeyear{Ashby:2013cc}) incorporating the pre-existing cryogenic observations from the GOODS Spitzer Legacy project (PI: M. Dickinson). Complementary to the space based imaging of HST and Spitzer is the ground-based imaging of the CTIO U band, VLT/VIMOS U band \citep{Nonino:2009hf}, VLT/ISAAC $K_{s}$ \citep{Retzlaff:2010co} and VLT/HAWK-I $K_{s}$ (Fontana et al. \emph{in prep.}) bands.

\subsection{Source photometry and deconfusion}
The full details on how the source photometry was obtained are outlined in \citet{Guo:2013ig}, however we provide a brief summary of the method used for reference here. Photometry for the HST bands was done using SExtractor's dual image mode, using the WFC3 H band mosaic as the detection image and the respective ACS/WFC3 mosaics as the measurement image after matching of the point-spread function (PSF). 

For the ground-based (VIMOS and CTIO U band and ISAAC and Hawk-I Ks) and Spitzer IRAC bands, deconvolution and photometry was done using template fitting photometry (TFIT). We refer the reader to \citet{Laidler:2007iy},\citet{2012ApJ...752...66L} and the citations within for further details of the TFIT process and the improvements gained on mixed wavelength photometry.


\subsection[Subsec]{Subsection}
\label{sec:subsec2_label}


%% %% End of file...  %%
