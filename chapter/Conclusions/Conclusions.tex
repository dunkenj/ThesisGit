\chapter{Conclusions and Future Prospects}\label{ch:conclusion}

In the introduction to this thesis I highlighted some of the outstanding questions in astronomy and the study of galaxy evolution. This thesis aimed to address two key issues at high redshift, firstly how and when galaxies assembled their stellar mass at high redshift. And secondly what role galaxies played in the epoch of reionization, specifically whether they are the sources responsible for powering the reionization of the intergalactic medium. 

In the following chapter I will outline what new conclusions can be drawn from the results of this thesis on these particular issues. I will also discuss the future prospects for solving these questions and the work which can be done to build upon the results of this thesis.

\section{Finding and studying high redshift galaxies}
In Section~\ref{sec:intro-distant-galaxies} of the thesis introduction, we outlined the evolution in recent years of the methods and techniques used to find and study galaxies at increasingly high redshifts. Chief among these methods for finding galaxies at $z>3$ are Lyman break galaxies (LBG) and photometric redshift analysis. Motivated by an apparent discrepancy between the observed colours of galaxies selected by these methods, in Chapter~\ref{ch:smf} we performed extensive tests on the use of photometric redshift selection at high redshift. We find that the large spread of colours for galaxies selected by photometric redshifts can be explained purely by the photometric scatter of faint galaxies (with the predicted intrinsic colours) out of the traditional colour criteria.

In addition, we conclude that photometric redshift selection can be much less sensitive to photometric scatter than Lyman break criteria for the same redshift range. Furthermore, it can also correctly identify high-redshift sources which would otherwise fail LBG selection criteria.

\section{The growth of stellar mass in the early universe}
In Chapter~\ref{ch:smf} we investigate the evolution of the galaxy stellar mass function (SMF) and star-formation rates at $z \geq 4$ in the CANDELS GOODS South field. Our new observations of the stellar mass function exhibit steep low-mass slopes, steeper than previously observed at these redshifts but in greater agreement with the theoretical predictions of semi-analytic and hydrodynamical models. Comparing the redshift evolution of the observed mass functions with those of the semi-analytic models illustrates their potential for constraining the physics of galaxy feedback in future analyses.

Our observed mass functions represent the first direct construction of the galaxy stellar mass function at $z\sim6$ and $z\sim7$. However, the large errors at these redshifts (due to the small number counts and large stellar mass errors) illustrates the need for both additional wide-area ground-based data and further ultra deep (and potentially lensed) observations to better constrain the high and low mass ends of the SMF respectively.

The steeper low-mass slopes measured in this work results in higher estimates of the stellar mass density than previously observed, even accounting for the effect of reduced stellar mass estimates due to the effects of nebular emission on SED fitting. For the high redshift samples studied in Chapter~\ref{ch:smf}, the observed stellar mass to UV luminosity ratios show minimal luminosity dependence and are consistent with a constant mass-to-light ratio at all redshifts. We also find that the overall normalisation of the mass-to-light distribution increases with redshift.

In addition to measuring the galaxy stellar masses, we also investigate the star-formation properties for our high-redshift galaxy samples. For a fixed stellar mass ($5~\times 10^{9}~\text{M}_{\odot}$), the average specific star-formation rate (sSFR) rises strongly with redshift, confirming recent observations and showing good agreement with the theoretical prediction that the sSFR is proportional to the specific accretion rate at early times. In combination with other recent studies, this result lays to rest the previous tension between theory and observation.

The star-formation rate densities observed in Chapter~\ref{ch:smf} are higher than previously measured at $z>3$, we find that this difference is most likely due to the differing treatment of dust extinction when correcting its effects. These differences highlight the large systematic uncertainties which still strongly affect the estimation of star-formation rates from the rest-frame ultra-violet emission of galaxies. In Section~\ref{sec:conc-future} of this chapter, we explore these systematic effects further and discuss how they could soon be addressed.

Another route through which galaxies can grow is via mergers with other galaxies in their local environment. However, due to the difficulty in estimating the merger rates of galaxies at $z\geq 3$, very few observational constraints on galaxy mergers have been made at these redshifts. In Chapter~\ref{ch:mergers}, we attempt to build on the existing merger constraints at high redshift by trialling a new method for measuring mergers in photometric surveys.

We apply a novel method for estimating the close pair statistics of galaxies to the CANDELS GOODS South field, studying the merger fraction at $z\geq 2$. In the redshift range of $2 \lesssim z \lesssim 4$, we find that for galaxies with $\log_{10}(\Mstar/\text{M}_{\odot}) > 9.5$ and merger ratios of 1:4 or less, the average timescale between major mergers ($\Gamma$) is roughly constant at $\approx 4$ Gyr. While for higher mass galaxies ($\log_{10}(\Mstar/\text{M}_{\odot}) > 10$) the average is $\approx 7$ to 10 Gyr over this same redshift range. Given these merger rates we conclude that in the few billion years preceding the peak in cosmic star-formation rate history, star-formation is the dominant mechanism of mass growth by approximately an order of magnitude. 

We find that the method used in Chapter~\ref{ch:mergers} produces robust results at the redshifts where there are good number statistics for galaxies above the completeness limits. However due to the compounding effects of mass completeness and decreasing number counts at high-redshift (as measured in Chapter~\ref{ch:smf}), the dataset used for this study is not sufficiently large enough to produce good results at $z\sim5$ and $z\sim6$. With the inclusion of the four remaining CANDELS fields in this analysis we are confident that robust estimates of the merger rate at $z \geq 5$ will soon be possible.

\section{The contribution of galaxies to reionization}
The epoch of reionization was the last major phase change of the Universe and represents one of the current frontiers of extragalactic astronomy. Of particular interest is the question of what sources powered this reionization process and whether star-forming galaxies are responsible. In Chapter~\ref{ch:reionization} we explore in detail the current constraints on the ionizing photon emissivity of galaxies during the epoch of reionization. As current observations strongly support a luminosity and redshift dependence for the UV continuum slope ($\beta$) during this period, we use extensive SED modelling to understand the effects of such evolution on the inferred ionizing emissivity.

A key conclusion of this chapter is that for evolution in any major galaxy property (e.g. age, metallicity, dust attenuation), as galaxies become bluer their ionizing efficiency increases. Or in other terms, any evolution in a property which makes a galaxy bluer, increases its ionizing efficiency. Since current observations strongly support a colour-magnitude relation where brighter galaxies are redder, this result has a significant effect on the inferred total ionizing emissivity at $z > 6$ and the potential for galaxies to power reionization. When assuming an ionizing efficiency which is dependent on the luminosity-weighted average $\beta$, the galaxy population which is brighter than current detection limits can produce enough ionizing photons to maintain reionization at $z\sim7$. 

The effect of the observed $\beta$ evolution on the ionizing emissivity is even more pronounced when the luminosity dependence is fully taken into account. Assuming an ionizing efficiency which is luminosity dependent increases the importance of faint galaxies due to the fact they are bluer, and hence more efficient at producing ionizing photons. While the predicted ionizing emissivity for the currently observed bright galaxies is reduced (due to their redder $\beta$), the increased efficiency of fainter galaxies means that only galaxies as faint as $M_{UV} \approx -15$ may be required to complete reionization at all redshifts. Galaxies as faint as this should be easily detected by the James Webb Space Telescope. We conclude that based on the existing colour constraints and observed star-formation densities, star-forming galaxies can easily produce enough ionizing photons to the power the reionization of hydrogen.

The model predictions of Chapter~\ref{ch:reionization} relate the observable properties to the fundamental but un-observable ionizing photon rates, and can therefore provide crucial testable predictions for the latest faint UV surveys. With mass, luminosity and colour dependent constraints on $f_{esc}$ at $2 \lesssim z \lesssim 3$, surveys such as the UVUDF \citep{Teplitz:2013jg} and GOODS UV Legacy Survey (PI: Oesch) will allow for significantly improved estimates of the ionizing emissivity during the epoch of reionization.

\section{Future prospects}\label{sec:conc-future}
One of the key conclusions we draw from the results of this thesis is that the key limitation in solving the outstanding questions in galaxy evolution at $z > 3$ is not the availability of suitable data, but actually the systematic uncertainties in some of the methods and assumptions used to analyse it. With the full five CANDELS HST fields now completed and the ongoing Frontier Fields survey providing even deeper observations over narrower areas, the potential available sample sizes are large enough to provide excellent number statistics at $z>3$. Furthermore, as shown by \citet{Bowler:vl,Bowler:2013wz}, deep ground-based surveys can provide vital constraints on the rarest and brightest sources not typically measured by the smaller HST fields. The addition of contiguous \emph{Spitzer} data (SPLASH, PI: Capak) means that similar analysis can also be extended to the high-mass end of stellar mass function at $z\sim 4$ and above.

As outlined in the introduction and proved by the results of this thesis, the spectral energy distributions (SEDs) of galaxies are an important tool for understanding galaxy evolution at high redshift. However, the information we can learn is limited not just by the quality of the data but also by our understanding and suitability of the ingredients which we put into the SED fitting.

The assumption of either a dust extinction curve (e.g. Milky Way or SMC-like) or an empirical attenuation curve (such as \citet{2000ApJ...533..682C} or \citet{2000ApJ...539..718C}) is a critical component in measuring and understanding the intrinsic SEDs of large galaxy samples. However, dust is arguably the least understood baryonic component of galaxies, especially at $z > 1$ where the observations of Herschel and ALMA are still only just beginning to be fully exploited. 

Although a universal attenuation law is widely assumed when fitting the SED of galaxies, it is well known that the attenuation law is not universal \citep{Buat:2012ku,Kriek:2013kx}. The assumption of a universal attenuation law to fit galaxy samples in which it systematically varies will bias estimates of the galaxy and stellar population properties we are trying to measure through SED fitting. Additionally, such variations will result in strong biases on the measurements of the UV continuum slope $\beta$ at high-redshift, leading to significantly biased estimates for dust-corrected star-formation rates. Measuring and understanding how the dust attenuation curve varies is essential to improving our physical understanding of the evolution of galaxies both at the epoch of peak star-formation and during their early formation at higher redshifts.

Using the latest generation of deep narrow-band \citep{PerezGonzalez:2012fo} and grism surveys \citep{Brammer:2012bu} combined with rest-frame far-infrared data (e.g. GOODS Herschel; \citet{Elbaz:2011ix}, or ALMA observations), it is now possible to fully quantify the systematic variation in dust attenuation curve at $1 \lesssim z \lesssim 4$ for the first time. For both bright individual galaxies and stacked samples of faint galaxies, it will be possible to measure the variation as a function of not just a galaxy’s stellar mass, luminosity and spectral type but also its morphology (both parametric and non-parametric). In addition to providing more reliable estimates of star-formation rates and SFR densities such as those calculated in Chapter~\ref{ch:smf}, a better understanding of dust attenuation is also vital in constraining the ionizing emissivity of galaxies during the epoch of reionization (see Chapter~\ref{ch:reionization}).

As well as the wealth of deep photometric data (CANDELS, Frontier Fields, UltraVISTA) which have yet to be utilised in full, the latest generation of spectroscopic instruments and surveys offer the potential for a much deeper understanding of galaxy properties at $z\sim3$ and beyond. Rest-frame optical spectroscopy from surveys such as MOSDEF \citep{Kriek:2014uw} can potentially put tight constraints on the ages and metallicities of galaxies at $z\sim3$. Similarly, the massively multiplexed spectroscopy of MUSE \citep{Bacon:2015eh} offers the potential for spatially resolved studies of Lyman-$\alpha$ emission out to redshifts of $z\sim6$. As well as helping to unravel the physical processes of galaxy formation at these redshifts, such observations will also serve as crucial priors for the study of galaxies deep into the epoch of reionization.

At longer wavelengths, the Atacama Large Millimetre Array (ALMA) has already shown in early science results that not only can it detect the gas and dust in distant galaxies, but actually study in detail the kinematics and structures of $z\sim5$ galaxies \citep{DeBreuck:2014eo}. The ALMA Deep Field (PI: Dunlop), a survey of the \emph{Hubble} Ultra Deep Field at a wavelength of 1.3mm is already underway and potentially offers constraints on the dust properties of galaxies at to $z\sim6$. At even longer wavelengths, the deep blank field surveys with LOFAR will allow the study of bright radio galaxies out to $z>5$ and even offer a new method for selecting high-redshift galaxies through their 21-cm forest. The era of \emph{truly} multi-wavelength studies of galaxies as far back as the epoch of reionization has most certainly begun.
